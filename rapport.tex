\documentclass{article}
\usepackage{blindtext}
\usepackage[utf8]{inputenc}

\usepackage{graphicx}
\graphicspath{ {./images/} }
\usepackage[font=small,labelfont=bf]{caption}
\usepackage{multirow}
\usepackage{array}
\usepackage{listings}
\usepackage{hyperref} 
\usepackage{caption}
\usepackage{subcaption}

\title{Rapport final - Projet ALN}
\date{25 Mai 2023}
\author{Mathis Verstrepen \and Matthieu Fontaine}

\begin{document}

\maketitle

\section{Introduction}

Toutes les fonctions nécessaires au calcul de l'ACP par la méthode historique et par la méthode moderne ont été implémentées. \\
Chaque fonction possède 3 tests unitaires différents. \\
Chaque méthode de calcul de l'ACP possède un test de bout en bout. \\
Chaque fonction est typée et documentée selon la convention de la docstring Python.

\section{ACP méthode historique}

\subsection{Hessenberg}

Nous avons programmé et testé notre version de la décomposition de Hessenberg. \\
Cette fonction nous permet d'obtenir la matrice tridiagonale nécessaire pour calculer l'ACP par la méthode historique. \\
Le paramètre $calc\_q$ a été implémenté.

Fichier~: \texttt{functions/eighvals/hessenberg.py}.

Commande de test~: \texttt{python3 test\_unitaire.py T1\_Hessenberg}.

\begin{small}
\begin{verbatim}
# Compute the Hessenberg decomposition of a square matrix

# Args:
# 	A (np.array): Matrix to decompose (must be square).
# 	calc_q (bool, optional): Whether to compute the orthogonal matrix Q. Defaults to False.

# Returns:
# 	np.array: Hessenberg matrix H.
# 	np.array: Orthogonal matrix Q. Only returned if calc_q is True.

def hessenberg(A: np.array, calc_q: bool = False) -> tuple[np.array, np.array]:
\end{verbatim}
\end{small}

\subsection{Gershgorin}

Nous avons programmé et testé notre version du calcul de l'intervalle des valeurs propres d'une matrice tridiagonale par la méthode des disques de Gershgorin. \\
Cette fonction nous permet d'obtenir l'intervalle des valeurs propres d'une matrice tridiagonale, nécessaire pour calculer par la suite les valeurs propres de cette même matrice. \\

Fichier~: \texttt{functions/eighvals/gershgoring\_trig.py}.

Commande de test~: \texttt{python3 test\_unitaire.py T2\_Gershgorin}.

\begin{small}
\begin{verbatim}
# Compute an interval containing all eigenvalues of a tridiagonal matrix

# Args:
#    C (np.array): Tridiagonal matrix.

# Returns:
#    np.float64: Lower bound of the interval, obtained via Gershgorin circle theorem
#    np.float64: Upper bound of the interval, obtained via Gershgorin circle theorem

def gershgorin_trig(C: np.array) -> tuple[np.float64, np.float64]:
\end{verbatim}
\end{small}

\subsection{Bissection}

Nous avons programmé et testé notre version du calcul des valeurs propres d'une matrice tridiagonale par la méthode de la bissection. \\

Fichier~: \texttt{functions/eighvals/bissection.py}.

Commande de test~: \texttt{python3 test\_unitaire.py T3\_Bissection}.

\begin{small}
\begin{verbatim}
# Compute the eigenvalues of a matrix A using the bissection method. 
# Recursively calls itself until the interval is small enough.

# Args:
#     A (np.array): Matrix to compute the eigenvalues of.
#     binf (np.float64): Inferior bound of the interval.
#     bsup (np.float64): Superior bound of the interval.
#     e (np.float32): Precision of the computation. Defaults to np.finfo(np.float32).eps 
#                     (machine precision in 32 bits).

# Returns:
#     np.array: Eigenvalues of A in the interval [binf, bsup]. 
#               Empty array if the interval does not contain any eigenvalue.

def bissection(A: np.array, binf: np.float64, bsup: np.float64, 
               precision: np.float32 = np.finfo(np.float32).eps) -> np.array:
\end{verbatim}
\end{small}

Cette fonction est dépendante de deux autres fonctions que nous avons également programmé et testé nous-même:

\subsubsection{Séquence de Sturm}

Nous avons programmé et testé notre version du calcul de la séquence de Sturm d'une matrice avec un décalage $\mu$ des diagonales. \\

Fichier~: \texttt{functions/eighvals/bissection.py}.

Commande de test~: \texttt{python3 test\_unitaire.py T4\_SturmSeq}.

\begin{small}
\begin{verbatim}
# Compute the Sturm sequence of a matrix A.

# Args:
#     A (np.array): Matrix to compute the Sturm sequence of.
#     mu (int): Adjusts the matrix A by subtracting mu from the diagonal.

# Returns:
#     np.array: Sturm sequence of A.

def sturm_seq(A: np.array, mu: int) -> np.array:
\end{verbatim}
\end{small}

\subsubsection{Comptage changement de signe}

Nous avons programmé et testé une fonction pour calculer le nombre de changements de signe dans une séquence de Sturm. \\
On suppose que la séquence de Sturm fournie commence toujours bien par 1.

Fichier~: \texttt{functions/eighvals/bissection.py}.

Commande de test~: \texttt{python3 test\_unitaire.py T5\_CountSignChange}.

\begin{small}
\begin{verbatim}
# Compute the number of sign changes in a Sturm sequence.

# Args:
#     seq (np.array): Sequence of determinants.

# Returns:
#     int: Number of sign changes.

def count_sign_changes(seq: np.array) -> int:
\end{verbatim}
\end{small}

\subsection{Puissance Inverse}

Nous avons programmé et testé notre version du calcul d'un vecteur propre associé à une valeur propre d'une matrice par la méthode de la puissance inverse. \\

Fichier~: \texttt{functions/eighvals/puissance\_inverse.py}.

Commande de test~: \texttt{python3 test\_unitaire.py T6\_puissance\_inverse}.

\begin{small}
\begin{verbatim}
# Compute the eigenvector of a matrix A using the inverse method and the given eigenvalue mu.

# Args:
#     A (np.array): Matrix to compute the eigenvector of.
#     mu (np.float64): One of the eigenvalues of A.
#     epsilon (np.float32): Precision of the computation.

# Returns:
#     np.array: Eigenvector of A corresponding to the eigenvalue mu.

def puissance_inverse(A: np.array, mu: np.float64, epsilon: np.float32 = np.finfo(np.float32).eps):
\end{verbatim}
\end{small}

Cette fonction est dépendante de deux autres fonctions que nous avons également nous-même programmé et testé :

\subsubsection{Solve Triangular}

Nous avons programmé et testé notre version de la fonction \texttt{solve\_triangular}. \\
Les paramètres lower, overwrite\_b et unit\_diagonal ont été implémentés.

Fichier~: \texttt{functions/eighvals/puissance\_inverse.py}.

Commande de test~: \texttt{python3 test\_unitaire.py T7\_solve\_triangular}.

\begin{small}
\begin{verbatim}
# Solve a linear system of equations with a triangular matrix.

# Args:
#     A (np.array): Triangular matrix. A from Ax = b.
#     b (np.array): Vector of constants. b from Ax = b.
#     lower (bool): If True, the matrix is lower triangular, otherwise it is upper triangular.
#     overwrite_b (bool): If True, the function overwrites the vector b.
#     unit_diagonal (bool): If True, the diagonal of the matrix is assumed to be 1. 
#                           Defaults to False.

# Returns:
#     np.array: Solution of the linear system of equations.

def solve_triangular(A: np.array, b: np.array, lower: bool, overwrite_b: bool, 
                     unit_diagonal: bool = False):
\end{verbatim}
\end{small}

\subsubsection{Factorisation LU}

Nous avons programmé et testé notre version de la décomposition LU d'une matrice carrée.

Fichier~: \texttt{functions/eighvals/puissance\_inverse.py}.

Commande de test~: \texttt{python3 test\_unitaire.py T8\_lu\_factor}.

\begin{small}
\begin{verbatim}
# Compute the LU factorization of a matrix A.

#    Args:
#        A (np.array): Matrix to factorize.

#    Returns:
#        tuple[np.array, np.array]: Tuple containing the factorized matrix and the pivot vector.

def lu_factor(A: np.array) -> tuple[np.array, np.array]:
\end{verbatim}
\end{small}

\subsection{Eigh}

Cette fonction est une "wrapper function", qui sert à exécuter facilement tout le déroulé d'obtention des valeurs propres et des vecteurs propres par les fonctions précédentes.

Fichier~: \texttt{functions/eighvals/eigh.py}.

Commande de test~: \texttt{python3 test\_unitaire.py T9\_eigh}.

\begin{small}
\begin{verbatim}
# Compute the eigenvalues and eigenvectors of a symmetric matrix A.
# Same as numpy.linalg.eigh

# Args:
#     A (np.array): Symmetric matrix to compute the eigenvalues and eigenvectors.

# Returns:
#     tuple[np.array, np.array]: Eigenvalues and eigenvectors of A.

def eigh( A : np.array ) -> tuple[np.array, np.array]:
\end{verbatim}
\end{small}

\section{ACP méthode moderne}

\subsection{Bidiagonalisation}

Nous avons programmé et testé notre version du calcul de la bidiagonalisation d'une matrice. \\

Fichier~: \texttt{functions/svd/bidiagonal.py}.

Commande de test~: \texttt{python3 test\_unitaire.py T10\_bidiagonal}.

\begin{small}
\begin{verbatim}
# Compute the bidiagonal decomposition of a matrix A.

# Args:
#     A (np.array): Matrix to decompose.

# Returns:
#     tuple[np.array, list[np.array], list[np.array]]: B, VL, VR such that A = VL * B * VR.

def bidiagonal(A: np.array) -> tuple[np.array, list[np.array], list[np.array]]:
\end{verbatim}
\end{small}

Cette fonction est dépendante d'une autre fonction :

\subsubsection{Reflecteur}

Cette fonction calcule la matrice de reflexion associé à un vecteur v.

Fichier~: \texttt{functions/svd/bidiagonal.py}.

Commande de test~: \texttt{python3 test\_unitaire.py T11\_reflecteur}.

\begin{small}
\begin{verbatim}
# Compute the reflection matrix associated to a vector v.

# Args:
#     p (int): Number of rows of the matrix.
#     v (np.array): Vector to compute the reflection matrix.

# Returns:
#     np.array: Reflection matrix associated to v.

def reflecteur(p: int, v: np.array) -> np.array:
\end{verbatim}
\end{small}

\subsubsection{SVD}

Nous avons programmé et testé notre propre version de la fonction \texttt{svd}. \\
A noter que ladite fonction utilise un intervalle légèrement moins précis que celui donné par Gershgorin: on peut voir
dans le code que l'intervalle est initialement celui fourni par Gershgorin, mais élargi de 0,0001 aux deux extrêmes.
Il s'agit d'une mesure de précaution prise à cause d'un bug qui semble se produire seulement sur certaines versions
de Python, qui conduisait à la disparition d'une des valeurs propres. \\

Fichier~: \texttt{functions/svd/svd.py}.

Commande de test (un seul test): \texttt{python3 test\_unitaire.py T12\_svd}.

\begin{small}
\begin{verbatim}
# Compute the singular values and singular vectors of a matrix A.

# Args:
#     A (np.array): Matrix to decompose.

# Returns:
#     tuple[np.array, np.array, np.array]: U, Sigma, Vt such that A = U * Sigma * Vt.

def svd(A: np.array) -> tuple[np.array, np.array, np.array]:
\end{verbatim}
\end{small}

Cette fonction est dépendante de la fonction reflecteur précisisée precédemment.

\section{Tests et Résultats}

Tout les tests mis en place sont effectués avec succès :

\begin{center} 
\includegraphics[width=1\columnwidth]{résultats_test.png}
\captionof{figure}{Test de toutes les fonctions décrites dans ce rapport.}
\end{center} 

Pour voir l'utilisation de ces fonctions sur le cas général de l'ACP appliqué aux iris, utilisez les programmes suivants:

\begin{itemize}
\item ACP méthode historique : \texttt{acp\_iris\_eigh.py}
\item ACP méthode moderne : \texttt{acp\_iris\_svd.py}
\end{itemize}

\begin{figure}
\centering
\begin{subfigure}{.5\textwidth}
  \centering
  \includegraphics[width=1\linewidth]{acp_iris_eigh_res.png}
  \caption{ACP méthode historique}
  \label{fig:sub1}
\end{subfigure}%
\begin{subfigure}{.5\textwidth}
  \centering
  \includegraphics[width=1\linewidth]{acp_iris_svd_res.png}
  \caption{ACP méthode moderne}
  \label{fig:sub2}
\end{subfigure}
\caption{Résultats des ACP effectuées avec nos fonctions}
\label{fig:test}
\end{figure}


Les résultats sont concluants et nous obtenons des graphiques similaires ou extrêmement proches de ceux produits par le fichier original fourni.

\end{document}
